\begin{theorem}[subestrutura ótima]
    Seja $n$ um inteiro com representação mínima \\$T = [t_0, t_1, \ldots, t_k]$, sendo $k > 0$. Então, $[t_1, \ldots, t_k]$ é uma representação mínima de $\dfrac{n - t_0}{3}$.
\end{theorem}

\begin{proof}[\textbf{Demonstração}]
    Considere $m = (n - t_0) / 3$ e sua representação mínima $R = [r_0, r_1, \ldots, r_{l^*}]$ de tamanho $l^*$. Suponha agora uma representação $S = [s_0, \ldots, s_l]$ de $m$ que não é ótima, isto é, $l > l^*$. Assim, temos que:
    \begin{align*}
        n &= 3 m + t_0 \\
        &= t_0 + 3 \left(r_0 + r_1 \cdot 3 + \cdots + r_{l^*} \cdot 3^l\right) \\
        &= t_0 + r_0 \cdot 3 + r_1 \cdot 3^2 + \cdots + r_{l^*} \cdot 3^{l^*+1}
    \end{align*}
    Então, temos a representação $R' = [t_0, r_0, \ldots, r_{l^*}]$ para $n$. Da mesma forma, temos que $S' = [t_0, s_0, \ldots, s_l]$ representando o mesmo número. Entretanto,
    \[
        \abs{R'} = \abs{R} + 1 = l^* + 2 < l + 2 = \abs{S} + 1 = \abs{S'}
    \]

    Ou seja, a representação $S'$ não é ótima. Portanto, pela contrapositiva, para qualquer representação $A$ de $n$, se ela for mínima, $A - [t_0]$ será uma representação mínima de $m$. Como $T$ é ótima, o teorema segue.
\end{proof}

\begin{theorem}[escolha gulosa]
    Seja $[t_0, t_1, \ldots, t_k]$ a representação mínima de $n$. Então, o primeiro dígito é $t_0 = n - 3 \lfloor (n + 1) / 3 \rfloor$.
\end{theorem}

\begin{proof}
    ...
\end{proof}

\begin{codebox}
    \Procname{$\proc{Mínimo-Dígitos}(n)$}
    \li \kw{se} $n <= 1$ \kw{então}
        \Do
    \li     \kw{retorna} $0$
        \End
    \li \kw{senão}
        \Do
    \li     $t0 \gets ((n + 1) \modulo 3) - 1$
    \li     \kw{retorna} $\proc{Mínimo-Dígitos}((n - t0) / 3)$
        \End
\end{codebox}

\[
    T(n) = T(n / 3) + \Theta(1) \in \Theta(\log(n))
\]
