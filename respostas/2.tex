\def\pmin{\ensuremath{p_{\min}}\xspace}
\def\pmax{\ensuremath{p_{\max}}\xspace}
\def\pinf{\ensuremath{p_{\mathrm{inf}}}\xspace}
\def\psup{\ensuremath{p_{\mathrm{sup}}}\xspace}

\begin{theorem}[subestrutura ótima]
    Sejam $k$ e $S = (s_1, \ldots, s_n)$ uma solução de desequilíbrio mínimo. Considere também que \pmin e \pmax são o mínimo e máximo de $P$. Então, para todo $1 \leq i \leq n$,
    \[
        s_i = \begin{cases}
            p_i + k & \text{se } \abs{p_i - \pmax} > \abs{p_i - \pmin} \\
            p_i - k & \text{se } \abs{p_i - \pmax} < \abs{p_i - \pmin} \\
            p_i \pm k & \text{caso contrário}
        \end{cases}
    \]
\end{theorem}

\itemdsep

\begin{theorem}[escolha gulosa]
    Seja $\pinf = \max\{p_i \mid \pmax - p_i > p_i - \pmin\}$ e $\psup = \min\{p_i \mid \pmax \leq p_i - \pmin\}$. Então, com
    \[
        k = \left\lceil\frac{\min\left\{\psup - \pmin, \pmax - \pinf\right\}}{2}\right\rceil
    \]

    Pode-se gerar uma sequência de desequilíbrio mínimo.
\end{theorem}

\itemdsep

\begin{codebox}
    \Procname{$\proc{Desequilíbrio-Mínimo}(P, n)$}
    \li $\id{p-min} \gets P[1]$
    \li $\id{p-max} \gets P[1]$
    \li \kw{para} $i \gets 2$ \kw{até} $n$
        \Do
    \li     $\id{p-min} \gets \min(\id{p-min}, P[i])$
    \li     $\id{p-max} \gets \max(\id{p-max}, P[i])$
        \End
    \li
    \li $\id{p-inf} \gets \id{p-min}$
    \li $\id{p-sup} \gets \id{p-max}$
    \li \kw{para} $i \gets 1$ \kw{até} $n$
        \Do
    \li     \kw{se} $\id{p-max} - P[i] > P[i] - \id{p-min}$ \kw{então}
            \Do
    \li         $\id{p-inf} \gets \max(\id{p-inf}, P[i])$
            \End
    \li     \kw{senão}
            \Do
    \li         $\id{p-sup} \gets \min(\id{p-sup}, P[i])$
            \End
        \End
    \li
    \li $\id{max-diff} \gets \min(\id{p-sup} - \id{p-min}, \id{p-max} - \id{p-inf})$
    \li \kw{retorna} $\lceil \id{max-diff} / 2 \rceil$
\end{codebox}

\itemdsep

Podemos ver que todos os laços na função executam no máximo $n$ vezes. Então, a complexidade de tempo do algoritmo é $T(n) \in \Theta(n)$. Como não é usado nenhum armazenamento adicional, o espaço tem complexidade constante, $E(n) \in \Theta(1)$.
