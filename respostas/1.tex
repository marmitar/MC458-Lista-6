A solução será dada por lista de cortes $C = [C_1, \ldots, C_n]$, em que cada elemento é uma lista $C_i = [c_{i, 0}, \ldots, c_{i, k}]$ de peças $0 \leq c_{i, j} \leq t_j$ tal que a soma dos comprimentos dos segmentos não ultrapasse o comprimento de um trilho, ou seja
\[
    \sum_{j = 0}^k 2^j \cdot c_{i, j} \leq M \qquad \text{ para todo } 1 \leq i \leq n
\]
Além disso, a lista como um todo deve suprir os segmentos necessários, então,
\[
    \sum_{i = 1}^n c_{i, j} = t_j \qquad \text{ para todo } 0 \leq j \leq k
\]

\begin{theorem}[subestrutura ótima]
    Seja $C = [C_1, C_2, \ldots, C_n]$ uma lista de cortes com número de trilhos $n$ mínimo. Então, a sublista $C / C_1 = [C_2, \ldots, C_n]$ tem número de trilhos mínimo para os segmentos não tratados em $C_1$.
\end{theorem}

\begin{theorem}[escolha gulosa]
    Seja $D = [d_0, \ldots, d_k]$ uma lista segmentos onde \[
        d_i = \max \left\{0 \leq q \leq t_i \midd q \cdot 2^i + \sum_{j = i + 1}^k d_j \cdot 2^j \leq M \right\}
    \] Se $D$ for não-vazia, então existe uma lista de cortes ótima que contém $D$.
\end{theorem}

\begin{proof}

\end{proof}

% \begin{codebox}
%     \Procname{$\proc{Mínimo-Trilhos}(t, k, M)$}
%     \li Seja $C$ um \red{bla bla bla}
%     \li $n \gets 0$
%     \li
%     \li $\id{max-i} \gets k$
%     \li \kw{enquanto} $\id{max-i} \geq 0$
%         \Do
%     \li     \kw{enquanto} $t[\id{max-i}] \isequal 0$
%     \li     $k \gets k + 1$
%     \li     \kw{se} $k \modulo 3 \isequal 0$ \kw{então}
%             \Do
%     \li         $n \gets n / 3$
%             \End
%     \li     \kw{senão}, \kw{se} $k \modulo 3 \isequal 1$ \kw{então}
%             \Do
%     \li         $n \gets (n - 1) / 3$
%             \End
%     \li     \kw{senão} ~ ~ ~ \Comment{$k \modulo 3 \isequal 2$}
%             \Do
%     \li         $n \gets (n + 1) / 3$
%             \End
%         \End
%     \li
%     \li \kw{retorna} $k$
% \end{codebox}
