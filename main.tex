\documentclass[a4paper, 14pt]{extarticle}

\usepackage[brazilian]{babel}
\usepackage[utf8]{inputenc}
\usepackage[T1]{fontenc}
\usepackage[margin=1.5cm,top=1.8cm,noheadfoot=true]{geometry}

\usepackage{float, pgf, caption, subcaption}

% \input{secoes}
% \input{teorema}
\input{simbolos}

% math display skip
\newcommand{\reducemathskip}[1][0.5em]{%
    \setlength{\abovedisplayskip}{1pt}%
    \setlength{\belowdisplayskip}{#1}%
    \setlength{\abovedisplayshortskip}{#1}%
    \setlength{\belowdisplayshortskip}{#1}%
}

% url linking problems
\def\url#1{\href{#1}{\texttt{#1}}}
% vermelho
\def\red#1{\textcolor{red}{#1}}

\def\lmref#1{\thmref[lema ]{#1}}

\usepackage{xparse, caption, booktabs}
\usepackage[hidelinks]{hyperref}
\usepackage[nameinlink, brazilian]{cleveref}
\crefformat{equation}{#2eq.~#1#3}
\crefformat{definition}{#2def.~#1#3}
\crefformat{proof}{#2dem.~#1#3}
\usepackage[section, newfloat]{minted}
\definecolor{sepia}{RGB}{252,246,226}
\setminted{
    bgcolor = sepia,
    % style   = pastie,
    frame   = leftline,
    autogobble,
    samepage,
    python3,
}
\setmintedinline{
    bgcolor={}
}

\theoremstyle{plain}
\newtheorem*{hypothesis}{Hipótese}
\newtheorem*{theorem}{Teorema}
\newtheorem*{hypothesisf}{Hipótese Fortalecida}

\newtheoremstyle{definicao}% name of the style to be used
  {}% measure of space to leave above the theorem. E.g.: 3pt
  {}% measure of space to leave below the theorem. E.g.: 3pt
  {}% name of font to use in the body of the theorem
  {}% measure of space to indent
  {\bf}% name of head font
  {:}% punctuation between head and body
  {.8em}% space after theorem head; " " = normal interword space
  {\thmnote{\textbf{#3}}}% Manually specify head
\theoremstyle{definicao}
\newtheorem*{definition}{Definição}

\NewDocumentCommand{\seq}{ s m O{n} O{\in\natural} }
    {\IfBooleanTF{#1}
        {\ensuremath{\left({#2}_{#3}\right)}}
        {\ensuremath{\left({#2}_{#3}\right)_{{#3}{#4}}}}}


\usepackage{titling, titlesec, enumitem}
% \usepackage{algorithmic}
\usepackage{clrscode3e, xspace}
\title{\vspace{-2.5cm}\Large Lista de Exercícios Avaliativa 6 \\ \normalsize MC458 - 2s2020 - Tiago de Paula Alves - 187679}
\preauthor{}\author{}\postauthor{}
\predate{}\date{}\postdate{}
\posttitle{\par\end{center}\vskip-1em}

\titleformat{\section}{\large\bfseries}{\thesection}{.8em}{}
\titlespacing*{\section}{0pt}{.5em plus .2em minus .2em}{.5em plus .2em}

\newlist{casos}{enumerate}{2}
\setlist[casos]{wide,labelwidth={\parindent},listparindent={\parindent},parsep={\parskip},topsep={0pt},label=\textbf{Caso \arabic*}:}
\setlist[casos,2]{label=\textbf{Caso \arabic{casosi}\alph*}:}

\newlist{ncasos}{description}{2}
\setlist[ncasos]{wide,listparindent={\parindent},parsep={\parskip},topsep={0pt}}

\titleformat{\section}[runin]
    {\titlerule{}\vspace{1ex}\normalfont\Large\bfseries}{}{1em}{}[.]
\titleformat{\subsection}[runin]
    {\normalfont\large\bfseries}{}{1em}{}[)]

% linha final da página ou seção
\newcommand{\docline}[1][\\]{%
    #1\noindent\rule{\textwidth}{0.4pt}%
    \pagebreak%
}
\newcommand{\itemdsep}{
    \noindent\hfil\rule{0.5\textwidth}{.2pt}\hfil
    \vskip1em
}


\usepackage{tikz}
\usetikzlibrary{calc,trees,positioning,arrows,fit,shapes,calc}

\DeclareMathSymbol{\mlq}{\mathord}{operators}{``}
\DeclareMathSymbol{\mrq}{\mathord}{operators}{`'}
\def\gets{~\leftarrow~}

\usepackage{fancyhdr}
\pagestyle{empty}

% \usepackage{showframe}
\begin{document}

    \maketitle
    \thispagestyle{empty}

    % \noindent\rule{\textwidth}{0.4pt}
    % \begin{center}\Large\vskip-0.5em
    %     CORRIGIR A QUESTÃO \textbf{???}.
    % \end{center}

    \section{1}
    \begingroup
        A solução será dada por uma lista $C = [C_1, \ldots, C_n]$, em que cada elemento é uma tupla $C_i = (c_{i, 0}, \ldots, c_{i, k})$ de peças $0 \leq c_{i, j} \leq t_j$ tal que a soma dos comprimentos dos segmentos não ultrapasse o comprimento de um trilho, ou seja,
\[
    \sum_{j = 0}^k 2^j \cdot c_{i, j} \leq M \qquad \text{ para todo } 1 \leq i \leq n
\]
Além disso, a lista como um todo deve suprir os segmentos necessários, então,
\[
    \sum_{i = 1}^n c_{i, j} = t_j \qquad \text{ para todo } 0 \leq j \leq k
\]

\begin{theorem}[subestrutura ótima]
    Seja $C = [C_1, C_2, \ldots, C_n]$ uma lista de cortes com número de trilhos $n$ mínimo. Então, a sublista $C / C_1 = [C_2, \ldots, C_n]$ tem número de trilhos mínimo para os segmentos não tratados em $C_1$.
\end{theorem}

\begin{theorem}[escolha gulosa]
    Seja $D = (d_0, \ldots, d_k)$ uma tupla de segmentos onde \[
        d_i = \max \left\{0 \leq q \leq t_i \midd q \cdot 2^i + \sum_{j = i + 1}^k d_j \cdot 2^j \leq M \right\}
    \]

    Se existir um elemento positivo em $D$, então existe um conjunto de cortes ótimo que contém $D$.
\end{theorem}

\begin{proof}

\end{proof}

\begin{codebox}
    \Procname{$\proc{Segmentos-Ótimos}(t, k, M)$}
    \li Seja $D[0, \ldots, k]$ um vetor.
    \li
    \li $positivo \gets \const{Falso}$
    \li \kw{para} $i \gets k$ \kw{descendo até} $0$
        \Do
    \li     $D[i] = \min\left(\lfloor M / 2^i \rfloor, t[i]\right)$
    \li     $M \gets M - D[i] \cdot 2^i$
    \li
    \li     \kw{se} $D[i] > 0$
            \Do
    \li         $positivo \gets \const{Verdadeiro}$
            \End
        \End
    \li
    \li \kw{se} $positivo$
        \Do
    \li     \kw{retorna} $D$
        \End
    \li \kw{senão}
        \Do
    \li     \kw{retorna} $\const{Nulo}$
        \End
\end{codebox}

\begin{codebox}
    \Procname{$\proc{Mínimo-Trilhos}(t, k, M)$}
    \li Seja $C$ uma lista vazia
    \li $n \gets 0$
    \li
    \li \kw{faça}
        \Do
    \li     $D \gets \proc{Segmentos-Ótimos}(t, k, M)$
    \li     \kw{se} $D \ne \const{Nulo}$
            \Do
    \li         $n \gets n + 1$
    \li         $C[n] \gets D$
    \li
    \li         \kw{para} $i = 0$ \kw{até} $k$
                \Do
    \li             $t[i] \gets t[i] - D[i]$
                \End
            \End
        \End
    \li \kw{enquanto} $D \ne \const{Nulo}$
    \li
    \li \kw{retorna} $(C, n)$
\end{codebox}

    \endgroup

    \docline[]

    \section{2}
    \begingroup
        \def\pmin{\ensuremath{p_{\min}}\xspace}
\def\pmax{\ensuremath{p_{\max}}\xspace}
\def\pinf{\ensuremath{p_{\mathrm{inf}}}\xspace}
\def\psup{\ensuremath{p_{\mathrm{sup}}}\xspace}

\begin{theorem}[subestrutura ótima]
    Sejam $k$ e $S = (s_1, \ldots, s_n)$ uma solução de desequilíbrio mínimo. Considere também que $\pmin < \pmax$ são o mínimo e máximo de $P$. Então, para todo $1 \leq i \leq n$,
    \[
        s_i = \begin{cases}
            p_i + k & \text{se } \abs{p_i - \pmax} > \abs{p_i - \pmin} \\
            p_i - k & \text{se } \abs{p_i - \pmax} < \abs{p_i - \pmin} \\
            p_i \pm k & \text{caso contrário}
        \end{cases}
    \]
\end{theorem}

\begin{proof}
    Suponha que exista um $i$ tal que $\abs{p_i - \pmax} > \abs{p_i - \pmin}$, mas $s_i = p_i - k$. Seja $S'$ a solução tal que $s'_i = p_i + k$ e $s'_j = s_j$ para todo $j \ne i$. Se $p_i = \pmin$, então $s_i$ é o menor valor de $S$.

    Nesse caso, já que $k \leq \left\lceil \pmax - \pmin / 2 \right\rceil$ para ser ótimo, temos que o mínimo de $S'$ deverá ser maior que o de $S$, sem mudar o máximo. Então, $D(S') < D(S)$, contradizendo a suposição inicial.

    Além disso, mesmo se $p_i > \pmin$, como ele está mais próximo do mínimo, o novo máximo em $S'$ terá um diferença menor que o mínimo, em relação a $S$. Isto é, $s'_{\max} - s_{\max} \leq s'_{\min} - s_{\min}$, ou seja, $D(S') \leq D(S)$, que é impossível.

    Logo, $s_i = p_i + k$. De forma similar, temos que $s_j = p_j - k$ quando $\abs{p_j - \pmax} < \abs{p_j - \pmin}$. Se $\abs{p_l - \pmax} = \abs{p_l - \pmin}$, isto é, $p_l = (\pmax + \pmin) / 2$, as duas soluções são válidas.
\end{proof}

\itemdsep

\begin{theorem}[escolha gulosa]
    Seja $\pinf = \max\{p_i \mid \pmax - p_i > p_i - \pmin\}$ e $\psup = \min\{p_i \mid \pmax \leq p_i - \pmin\}$. Então, com
    \[
        k = \left\lceil\frac{\min\left\{\psup - \pmin, \pmax - \pinf\right\}}{2}\right\rceil
    \]

    Pode-se gerar uma sequência de desequilíbrio mínimo.
\end{theorem}

\begin{proof}
    Para valores pequenos de $k$, $s_{\mathrm{min}} = \pmin + k$ continua sendo o mínimo da sequência e $s_{\mathrm{max}} = \pmax - k$, o máximo. No entanto, quando $k > (\psup - \pmin) / 2$ e $k > (\pmax - \pinf) / 2$, o mínimo se torna $s_{\mathrm{sup}} = \psup - k$ e o máximo, $s_{\mathrm{inf}} = \pinf + k$.

    Então, na faixa de transição, em que
    \[
        \frac{\psup - \pmin}{2} \leq k \leq \frac{\pmax - \pinf}{2}
    \]

    Ou
    \[
        \frac{\pmax - \pinf}{2} \leq k \leq \frac{\psup - \pmin}{2}
    \]

    Pode-se alcançar o desequilíbrio mínimo. Desse forma, o $k$ escolhido como \\$\left\lceil (\min\left\{\psup - \pmin, \pmax - \pinf\right\})/2\right\rceil$ é semrpe válido.
\end{proof}

\itemdsep

\begin{codebox}
    \Procname{$\proc{Desequilíbrio-Mínimo}(P, n)$}
    \li $\id{p-min} \gets P[1]$
    \li $\id{p-max} \gets P[1]$
    \li \kw{para} $i \gets 2$ \kw{até} $n$
        \Do
    \li     $\id{p-min} \gets \min(\id{p-min}, P[i])$
    \li     $\id{p-max} \gets \max(\id{p-max}, P[i])$
        \End
    \li
    \li $\id{p-inf} \gets \id{p-min}$
    \li $\id{p-sup} \gets \id{p-max}$
    \li \kw{para} $i \gets 1$ \kw{até} $n$
        \Do
    \li     \kw{se} $\id{p-max} - P[i] > P[i] - \id{p-min}$ \kw{então}
            \Do
    \li         $\id{p-inf} \gets \max(\id{p-inf}, P[i])$
            \End
    \li     \kw{senão}
            \Do
    \li         $\id{p-sup} \gets \min(\id{p-sup}, P[i])$
            \End
        \End
    \li
    \li $\id{max-diff} \gets \min(\id{p-sup} - \id{p-min}, \id{p-max} - \id{p-inf})$
    \li \kw{retorna} $\lceil \id{max-diff} / 2 \rceil$
\end{codebox}

\itemdsep

Podemos ver que todos os laços na função executam no máximo $n$ vezes. Então, a complexidade de tempo do algoritmo é $T(n) \in \Theta(n)$. Como não é usado nenhum armazenamento adicional, o espaço tem complexidade constante, $E(n) \in \Theta(1)$.

    \endgroup

    \docline[]

    \section{3}
    \begingroup
        \begin{theorem}[subestrutura ótima]
    Seja $n$ um inteiro com representação mínima $\langle t_0, t_1, \ldots, t_k \rangle$, sendo $k > 0$. Então, $\langle t_1, \ldots, t_k \rangle$ é uma representação mínima de $\dfrac{n - t_0}{3}$.
\end{theorem}

\begin{proof}[\textbf{Demonstração}]
    Considere $m = (n - t_0) / 3$ e sua representação mínima $R = \langle r_0, r_1, \ldots, r_{l^*} \rangle$ de tamanho $l^*$. Suponha agora uma representação $S = \langle s_0, \ldots, s_l \rangle$ de $m$ que não é ótima, isto é, $l > l^*$. Assim, temos que:
    \begin{align*}
        n &= 3 m + t_0 \\
        &= t_0 + 3 \left(r_0 + r_1 \cdot 3 + \cdots + r_{l^*} \cdot 3^l\right) \\
        &= t_0 + r_0 \cdot 3 + r_1 \cdot 3^2 + \cdots + r_{l^*} \cdot 3^{l^*+1}
    \end{align*}
    Então, temos a representação $R' = \langle t_0, r_0, \ldots, r_{l^*} \rangle$ para $n$. Da mesma forma, temos que $S' = \langle t_0, s_0, \ldots, s_l \rangle$ representando o mesmo número. Entretanto,
    \[
        \abs{R'} = \abs{R} + 1 = l^* + 2 < l + 2 = \abs{S} + 1 = \abs{S'}
    \]

    Ou seja, a representação $S'$ não é ótima. Portanto, pela contrapositiva, para qualquer representação $A$ de $n$, se ela for mínima, $A - \langle t_0 \rangle$ será uma representação mínima de $m$. Como $\langle t_0, t_1, \ldots, t_k \rangle$ é ótima, o teorema segue.
\end{proof}

\itemdsep

\begin{theorem}[escolha gulosa]
    Seja $\langle t_0, \ldots, t_k \rangle$ uma representação mínima de $n$. Então, o primeiro dígito é $t_0 = (n + 1) \bmod 3 - 1$.
\end{theorem}

\begin{proof}
    Seja $r = (n + 1) \bmod 3 - 1$ e note que
    \[
        n - t_0 = \sum_{i = 0}^k t_i \cdot 3^i - t_0 = 3 \left(\sum_{i = 1}^k t_i \cdot 3^{i-1} \right)
    \]
    Como $t_i \in \integer$ e $3^{i - 1} \in \integer$ para $i > 0$, então $n - t_0 \equiv 0 \dmod{3}$. Portanto, $n \equiv t_0 \dmod{3}$.

    ~

    \begin{casos}
        \item $n \equiv 0 \dmod{3}$. Então, $(n + 1) \equiv 1 \dmod{3}$ e $r = 0$. Como $-1 \not\equiv 0 \dmod{3}$ e $1 \not\equiv 0 \dmod{3}$, temos que $t_0 = 0 = r$.
        \item $n \equiv 1 \dmod{3}$. De forma similar, temos que $t_0 = 1 = r$.
        \item $n \equiv 2 \dmod{3}$. Note que $(n + 1) \equiv 0 \dmod{3}$, então $r = -1 = t_0$.
    \end{casos}
\end{proof}

\itemdsep

\begin{codebox}
    \Procname{$\proc{Mínimo-Dígitos}(n)$}
    \li \kw{se} $n \isequal 0$ \kw{então}
        \Do
    \li     \kw{retorna} $(0, [0])$
        \End
    \li
    \li Seja $T$ uma lista vazia
    \li $k \gets 0$
    \li \kw{enquanto} $n > 0$ \kw{faça}
        \Do
    \li     $T[k] \gets ((n + 1) \bmod 3) - 1$
    \li     $n \gets (n - T[k]) / 3$
    \li     $k \gets k + 1$
        \End
    \li
    \li \kw{retorna} $(k, T)$
\end{codebox}

\itemdsep

Pelo laço da função $\proc{Mínimo-Dígitos}$, temos que a complexidade de tempo é $T(n) \in \Theta(k)$. Com $k$ dígitos, o menor número representável é com $t_i = -1$ e o maior é $t_i = +1$, para $i < k$. Para que $n$ seja positivo e $T$ seja a representação mínima, é necessário que $t_k = 1$. Logo,
\[
    3^{k - 1} \leq \frac{3^k + 1}{2} = 3^k - \sum_{i = 0}^{k - 1} 3^i \leq n \leq \sum_{i = 0}^k 3^k = \frac{3^{k + 1} - 1}{2} \leq 3^{k + 1}
\]
Assim, podemos limitar $k$ por $\log_3 n - 1 \leq k \leq \log_3 n + 1$. Portanto, $T(n) \in \Theta(\lg n)$. Devido ao armazenamento do resultado, a complexidade de espaço também é $\Theta(k) = \Theta(\lg n)$.

    \endgroup

    \docline[]

\end{document}
